%        File: Modeling_the_Formation_of_Rain.tex

\documentclass[twocolumn,a4paper,10pt]{article}
\usepackage{times}
\usepackage{fullpage}
\usepackage{amsmath}
\usepackage{amssymb}
\usepackage{graphicx}
\usepackage[usenames,dvipsnames]{color}
\usepackage{subfig}
\usepackage{wrapfig}
\usepackage{gensymb}
\usepackage{array}
\usepackage{listings}
\usepackage{titling}
\usepackage{minted}
\definecolor{LightGray}{rgb}{0.95,0.95,0.95}
\newminted{c}{
  linenos,
  frame=single,
  fontsize=\footnotesize,
  bgcolor=LightGray,
}
% Hyperref must be last
\usepackage[backref,colorlinks,linkcolor=blue]{hyperref}

\newcommand{\TODO}{{\huge\emph{\color{red}!}}}

\title{Modeling the Formation of Rain using Hi-Performance Erlang}
\author{John Tyree\\
University of Amsterdam\\
\texttt{tyree@science.uva.nl}}
\date{\today}
\hypersetup{
  pdftitle={\thetitle},
  pdfauthor={John Tyree}
}

\begin{document}
\maketitle
\section{Abstract}
\section{Introduction}
Most computationally intensive simulations are made using a combination of C,
C++, and FORTRAN in the interest of speed. Technological innovation in the field
of single core processors is no longer following moore's law and has stalled
around 3Ghz. This hasn't meant the halt of progress in computing power, though,
as the industry turns to parallelism to acheive its gains. While frameworks
for developing software to utilize multiple computational threads exist in these
common languages, newer languages are being developed which may ultimately be
more suited to this style of programming. This paper explores the efficacy of
Erlang as a new computational tool. As a test case, a simulation of the
formation of rain in a cloud is presented.

\section{Theory of Rain Formation}
Discuss the progressional development of rain from condensation on a dust speck
to condensation of multidrops together to actual rain sized droplets falling.
Can also discuss splitting and rejoining of droplets. Talk about updrafting
suspending particles in the air until they have time to come together.




\end{document}


